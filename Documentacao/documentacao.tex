\documentclass[10pt,technote]{IEEEtran}

\usepackage{cite}
\usepackage{amsmath,amssymb,amsfonts}
\usepackage{graphicx}
\usepackage{textcomp}
\usepackage{xcolor}

\usepackage{blindtext}
\usepackage[brazilian]{babel}
\usepackage[utf8]{inputenc}
\usepackage[T1]{fontenc}

\def\BibTeX{{\rm B\kern-.05em{\sc i\kern-.025em b}\kern-.08em
    T\kern-.1667em\lower.7ex\hbox{E}\kern-.125emX}}
\begin{document}
\title{Toonify:\\
{\footnotesize \textsuperscript{*}a project in image processing}
}
\author{\IEEEauthorblockN{1\textsuperscript{st} Ariel Vieira Lima Serafim}
\IEEEauthorblockA{\textit{UNIVERSITY OF BRASÍLIA} \\
\textit{DEPARTMENT OF COMPUTER SCIENCE}\\
Brasília, Brazil \\
arielserafim@gmail.com\\}
\and
\IEEEauthorblockN{2\textsuperscript{nd} Luigi Minardi Ferreira Maia}
\IEEEauthorblockA{\textit{UNIVERSITY OF BRASÍLIA} \\
\textit{DEPARTMENT OF COMPUTER SCIENCE}\\
Brasília, Brazil \\
luigimaia99@gmail.com}
}

\maketitle
\begin{abstract}
This project is an attempt to reproduce and improve the algorithm proposed by Kevin Dade in his article\cite{CITE-artigobase}
\textit{Toonify: Cartoon Photo Effect Application}, that seeks to emulate the types of cel-sheding
\end{abstract}

\section{Introdução}
	O efeito toon é muito conhecido, trata-se de tornar objetos, seres e paisagens reais mais semelhantes à desenhos do estilo cartoon, para isso é preciso investir em alguma forma de simplificar as imagens.\\
	A forma de simplificação utilizada aqui é uma adaptação  do processo de Cel shading\cite{CITE-celshading} para fotos.  
	
\section{Referências e trabalhos semelhantes}

	A base para o trabalho foi o artigo de Kevin Dade, Toonify: Cartoon Photo Effect Application \cite{CITE-artigobase}.
	
\section{Soluções propostas}
	Como objetivo do processo de cartoonificação das images é simplificar tanto as bordas quanto as cores da imagem, o processo é feito em três partes: processamento das bordas da imagem, processamento das cores da imagem e, no final, junta-se o resultado dos dois processos anteriores em apenas uma imagem. 
	
	Como o processo de adquirir as bordas da imagem e o de adquirir as cores são mutualmente independes, essas duas partes são feitas em paralelo a fim de economizar tempo de execução.
	
	\subsection{Adquirir uma imagem contendo as bordas}
		\subsubsection{Filtro mediana}
			O primeiro processamento feito sobre a imagem é um filtro de mediana com uma matriz 7x7 como kernel, esse passo é importante para diminuir possíveis ruídos salt and pepper e também para diminuir o número de falsas bordas que poderiam ser detectadas caso a imagem estivesse detalhada demais.Vale lembrar que o tamanho do kernel garante a manutenção de detalhes relevantes.
			
		\subsubsection{Detector de bordas Canny}
			Esta é a etapa principal da parte de detecção de bordas. Outros algoritmos poderiam ter sido utilizados, como o algoritmo laplaciano, entretanto outras operações teriam de ser feitas sobre o resultado pois o laplaciano não garante que as bordas tenham largura 1 e isso gera efeitos indesejáveis no resultado final, além de dificultar o uso de operações morfológicas descritas a seguir.
			
		\subsubsection{Operações morfologicas}
			Obtidas as bordas da imagem, é necessário ainda que sejam feitos processamentos morfologicos para melhorar o efeito das bordas na imagem final. A operação escolhida aqui é a dilatação, esse processo realça as bordas e tende a juntar regiões, o que é importante para a eficiência do passo seguinte.
			
		\subsubsection{Filtrar bordas}
			Nesta etapa buscamos remover bordas muito pequenas, que geram um efeito muito semelhante ao ruído salt and pepper. Para remover esse "ruído" interpretamos as bordas como regiões conectadas e removemos as que possuirem uma área menor que um limite determinado.
		
		\subsection{Filtrar as cores da imagem}
			\subsection{Adquirir o grau de "colorido" da imagem}
				Como cada imagem possui um aspecto único, filtrar diferentes imagens, sempre com os mesmos parâmetros pode significar priorizar o processamento de algumas imagens em relação a outras. O modo encontrado de diferenciar as imagens foi encontrando o seu grau de colorido por meio do processo descrito em \textit{Measuring colourfulness in natural images} \cite{CITE-colourfuness}.
			\subsubsection{Filtrar bilateralmente}
				A partir da imagem de origem, utiliza-se um processo conhecido por borrar as cores da imagem, porém mantendo as bordas bem definidas. Como, descrito no artigo \textit{A Gentle Introduction to Bilateral Filtering and its Applications} \cite{CITE-bilateralfilter}.
				Como o filtro tem um grau de complexidade grande, opta-se por diminuir a imagem à metade e aplicar o filtro iterativamente com um kernel não tão grande de tamanho 9 por 9. O filtro possui duas entradas que são as variâncias $\sigma_r^2$ utilizadas pela gaussiana para cor e $\sigma_s^2$ que é a variância utilizada pelo filtro para distância entre os pixels. Além das entradas do filtro, é necessário escolher $I$ que é quantas vezes o filtro será utilizado iterativamente. Com alguns testes, decidiu-se escolher as essas constantes da seguinte forma:
				$$
				\sigma_r^2 = 	\frac{685}{C}
				$$
				$$
				\sigma_s^2 = 5
				$$
				$$
				I = 10 + \frac{C}{10}
				$$
				em que $C$ é o grau de colorido da imagem.
				No final, volta-se a imagem para o tamanho de origem.
			\subsubsection{Filtrar com a mediana}
			Neste estágio, deve-se utilizar o filtro da mediana para olcutar qualquer erro que tenha ocorrido durante o aumento da imagem para o tamanho original. Utilizou-se um kernel de tamaho 7 por 7.
			\subsubsection{Requantificar as cores}
			Para que o número de cores fosse reduzido, realiza-se sobre cada canal dos pixels uma operação de requantização que em dado pela seguinte fórmula:
			$$
			p_{new} = \left\lfloor\frac{p}{a}\right\rfloor \cdot a
			$$
			em que $p$ é o valor do canal do pixel e $a$ é um fator de pelo qual o número de cores possíveis será diminuido. Porém, a aplicação desse efeito tem como aspecto negativo escurecer as imagens, então, optou-se pelo uso da seguinte fómula:
			$$
			p_{new} = \left\lfloor\frac{p}{a}\right\rfloor \cdot a
			+ 255 - \left\lfloor\frac{255}{a}\right\rfloor \cdot a
			$$
			em que 255 é o valor máximo das cores em cada canal de cores. Já o valor de $a$ é escolhido a partir da seguinte fórmula:
			$$
			a = 10 + \frac{C}{5}
			$$
			em que novamente $C$ é o grau de "colorido" da imagem.
			
			\section{•}
							
\section{Resultados experimentais}

\section{Conclusões}

\begin{thebibliography}{20}
	\bibitem{CITE-artigobase}
	Kevin Dade:\\
	Toonify: Cartoon Photo Effect Application.
	\bibitem{CITE-celshading}
	Cel shading\\
	https://en.wikipedia.org/wiki/Celshading
	\bibitem{CITE-colourfuness}
	David Hasler a and Sabine Süsstrunk\\
	Measuring colourfulness in natural images
	\bibitem{CITE-bilateralfilter}
	Sylvain Paris$^1$, Pierre Kornprobst$^2$, Jack Tumblin$^3$ and Frédo Durand$^4$\\
	A Gentle Introduction to Bilateral Filtering and its Applications
	


\end{thebibliography}

\end{document}
\grid
