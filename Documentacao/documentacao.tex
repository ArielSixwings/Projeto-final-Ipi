\documentclass[10pt,technote]{IEEEtran}

\usepackage{cite}
\usepackage{amsmath,amssymb,amsfonts}
\usepackage{graphicx}
\usepackage{textcomp}
\usepackage{xcolor}

\usepackage{blindtext}
\usepackage[brazilian]{babel}
\usepackage[utf8]{inputenc}
\usepackage[T1]{fontenc}

\def\BibTeX{{\rm B\kern-.05em{\sc i\kern-.025em b}\kern-.08em
    T\kern-.1667em\lower.7ex\hbox{E}\kern-.125emX}}
\begin{document}
\title{Toonify:\\
{\footnotesize \textsuperscript{*}a project in image processing}
}
\author{\IEEEauthorblockN{1\textsuperscript{st} Ariel Vieira Lima Serafim}
\IEEEauthorblockA{\textit{UNIVERSITY OF BRASÍLIA} \\
\textit{DEPARTMENT OF COMPUTER SCIENCE}\\
Brasília, Brazil \\
arielserafim@gmail.com\\}
\and
\IEEEauthorblockN{2\textsuperscript{nd} Luigi Minardi Ferreira Maia}
\IEEEauthorblockA{\textit{UNIVERSITY OF BRASÍLIA} \\
\textit{DEPARTMENT OF COMPUTER SCIENCE}\\
Brasília, Brazil \\
email address}
}

\maketitle
\begin{abstract}
This project is an attempt to reproduce and improve the algorithm proposed by Kevin Dade in his article\cite{CITE-artigobase}
\textit{Toonify: Cartoon Photo Effect Application}, that seeks to emulate the types of cel-sheding
\end{abstract}

\section{Introdução}
	O filme \textit{"Who Framed Roger Rabbit"}\cite{CITE-filme} (1988),foi responsável por aumentar o interesse em misturar desenhos e realidade. \\O efeito toon é muito conhecido, trata-se de tornar objetos, seres e paisagens reais mais semelhantes a desenhos do estilo cartoon.\\
\section{Referências e trabalhos semelhantes}
	A base para o trabalho foi o artigo \cite{CITE-artigobase}
\section{Soluções propostas}
\section{Resultados experimentais}
\section{Conclusões}

\begin{thebibliography}{20}
	\bibitem{CITE-artigobase}
	Kevin Dade:\\
	Toonify: Cartoon Photo Effect Application.
	\bibitem{CITE-filme}
	Who Framed Roger Rabbit\\
	https://en.wikipedia.org/wiki/Who\_Framed\_Roger\_Rabbit.
\end{thebibliography}

\end{document}
