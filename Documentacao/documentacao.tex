\documentclass[10pt,technote]{IEEEtran}

\usepackage{cite}
\usepackage{amsmath,amssymb,amsfonts}
\usepackage{graphicx}
\usepackage{textcomp}
\usepackage{xcolor}

\usepackage{blindtext}
\usepackage[brazilian]{babel}
\usepackage[utf8]{inputenc}
\usepackage[T1]{fontenc}

\def\BibTeX{{\rm B\kern-.05em{\sc i\kern-.025em b}\kern-.08em
    T\kern-.1667em\lower.7ex\hbox{E}\kern-.125emX}}
\begin{document}
\title{Toonify:\\
{\footnotesize \textsuperscript{*}a project in image processing}
}
\author{\IEEEauthorblockN{1\textsuperscript{st} Ariel Vieira Lima Serafim}
\IEEEauthorblockA{\textit{UNIVERSITY OF BRASÍLIA} \\
\textit{DEPARTMENT OF COMPUTER SCIENCE}\\
Brasília, Brazil \\
arielserafim@gmail.com\\}
\and
\IEEEauthorblockN{2\textsuperscript{nd} Luigi Minardi Ferreira Maia}
\IEEEauthorblockA{\textit{UNIVERSITY OF BRASÍLIA} \\
\textit{DEPARTMENT OF COMPUTER SCIENCE}\\
Brasília, Brazil \\
email address}
}

\maketitle
\begin{abstract}
This project is an attempt to reproduce and improve the algorithm proposed by Kevin Dade in his article\cite{CITE-artigobase}
\textit{Toonify: Cartoon Photo Effect Application}, that seeks to emulate the types of cel-sheding
\end{abstract}

\section{Introdução}
	O efeito toon é muito conhecido, trata-se de tornar objetos, seres e paisagens reais mais semelhantes à desenhos do estilo cartoon, para isso é preciso investir em alguma forma de simplificar as imagens.\\
	A forma de simplificação utilizada aqui é uma adaptação  do processo de Cel shading\cite{CITE-celshading} para fotos.  
\section{Referências e trabalhos semelhantes}
	A base para o trabalho foi o artigo \cite{CITE-artigobase}
\section{Soluções propostas}
	O processo foi feito em duas partes, descritas a seguir.
	\subsection{bordas}
		\subsubsection{Filtro mediana}
			O primeiro processamento feito sobre a imagem é um filtro de mediana com uma matriz 7x7 como kernel, esse passo é importante para diminuir possíveis ruídos salt and pepper e também para diminuir o número de falsas bordas que poderiam ser detectadas caso a imagem estivesse detalhada demais.Vale lembrar que o tamanho do kernel garante a manutenção de detalhes relevantes.\\
		\subsubsection{Detector de bordas Canny}
			Esta é a etapa principal da parte de detecção de bordas. Outros algoritmos poderiam ter sido usados, como o algoritmo laplaciano, entretanto outras operações teriam de ser feitas sobre o resultado pois o laplaciano não garante que as bordas tenham largura 1 e isso gera efeitos indesejáveis no resultado final, além de dificultar o uso de operações morfológicas descritas a seguir.\\
		\subsubsection{Operações morfologicas}
			Com as bordas da imagem em mãos, processamentos morfologicos são feitos para melhorar o efeito das bordas na imagem final.A operação escolhida aqui é a dilatação, esse processo realça as bordas e tende a juntar regiões, oque  é importante para a eficiência do passo seguinte.\\
		\subsubsection{Filtrar bordas}
			Nesta etapa buscamos remover bordas muito pequenas, que geram um efeito muito semelhante ao ruído salt and pepper. Para remover esse "ruído" interpretamos as bordas como regiões conectadas e removemos as que possuirem uma área menor que um limite determinado.  
\section{Resultados experimentais}

\section{Conclusões}

\begin{thebibliography}{20}
	\bibitem{CITE-artigobase}
	Kevin Dade:\\
	Toonify: Cartoon Photo Effect Application.
	\bibitem{CITE-celshading}
	Cel shading\\
	https://en.wikipedia.org/wiki/Celshading


\end{thebibliography}

\end{document}
